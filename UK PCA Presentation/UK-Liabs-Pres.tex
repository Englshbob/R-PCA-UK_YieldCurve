\PassOptionsToPackage{unicode=true}{hyperref} % options for packages loaded elsewhere
\PassOptionsToPackage{hyphens}{url}
%
\documentclass[ignorenonframetext,]{beamer}
\usepackage{pgfpages}
\setbeamertemplate{caption}[numbered]
\setbeamertemplate{caption label separator}{: }
\setbeamercolor{caption name}{fg=normal text.fg}
\beamertemplatenavigationsymbolsempty
% Prevent slide breaks in the middle of a paragraph:
\widowpenalties 1 10000
\raggedbottom
\setbeamertemplate{part page}{
\centering
\begin{beamercolorbox}[sep=16pt,center]{part title}
  \usebeamerfont{part title}\insertpart\par
\end{beamercolorbox}
}
\setbeamertemplate{section page}{
\centering
\begin{beamercolorbox}[sep=12pt,center]{part title}
  \usebeamerfont{section title}\insertsection\par
\end{beamercolorbox}
}
\setbeamertemplate{subsection page}{
\centering
\begin{beamercolorbox}[sep=8pt,center]{part title}
  \usebeamerfont{subsection title}\insertsubsection\par
\end{beamercolorbox}
}
\AtBeginPart{
  \frame{\partpage}
}
\AtBeginSection{
  \ifbibliography
  \else
    \frame{\sectionpage}
  \fi
}
\AtBeginSubsection{
  \frame{\subsectionpage}
}
\usepackage{lmodern}
\usepackage{amssymb,amsmath}
\usepackage{ifxetex,ifluatex}
\usepackage{fixltx2e} % provides \textsubscript
\ifnum 0\ifxetex 1\fi\ifluatex 1\fi=0 % if pdftex
  \usepackage[T1]{fontenc}
  \usepackage[utf8]{inputenc}
  \usepackage{textcomp} % provides euro and other symbols
\else % if luatex or xelatex
  \usepackage{unicode-math}
  \defaultfontfeatures{Ligatures=TeX,Scale=MatchLowercase}
\fi
% use upquote if available, for straight quotes in verbatim environments
\IfFileExists{upquote.sty}{\usepackage{upquote}}{}
% use microtype if available
\IfFileExists{microtype.sty}{%
\usepackage[]{microtype}
\UseMicrotypeSet[protrusion]{basicmath} % disable protrusion for tt fonts
}{}
\IfFileExists{parskip.sty}{%
\usepackage{parskip}
}{% else
\setlength{\parindent}{0pt}
\setlength{\parskip}{6pt plus 2pt minus 1pt}
}
\usepackage{hyperref}
\hypersetup{
            pdftitle={A PCA Approach to Hedging UK Pension Liabilities},
            pdfauthor={John St-Hill},
            pdfborder={0 0 0},
            breaklinks=true}
\urlstyle{same}  % don't use monospace font for urls
\newif\ifbibliography
\usepackage{longtable,booktabs}
\usepackage{caption}
% These lines are needed to make table captions work with longtable:
\makeatletter
\def\fnum@table{\tablename~\thetable}
\makeatother
\usepackage{graphicx,grffile}
\makeatletter
\def\maxwidth{\ifdim\Gin@nat@width>\linewidth\linewidth\else\Gin@nat@width\fi}
\def\maxheight{\ifdim\Gin@nat@height>\textheight\textheight\else\Gin@nat@height\fi}
\makeatother
% Scale images if necessary, so that they will not overflow the page
% margins by default, and it is still possible to overwrite the defaults
% using explicit options in \includegraphics[width, height, ...]{}
\setkeys{Gin}{width=\maxwidth,height=\maxheight,keepaspectratio}
\setlength{\emergencystretch}{3em}  % prevent overfull lines
\providecommand{\tightlist}{%
  \setlength{\itemsep}{0pt}\setlength{\parskip}{0pt}}
\setcounter{secnumdepth}{0}

% set default figure placement to htbp
\makeatletter
\def\fps@figure{htbp}
\makeatother


\title{A PCA Approach to Hedging UK Pension Liabilities}
\author{John St-Hill}
\date{06/10/2020}

\begin{document}
\frame{\titlepage}

\begin{frame}{UK DB Pension Market}
\protect\hypertarget{uk-db-pension-market}{}

\begin{itemize}
\tightlist
\item
  10 Million Members
\item
  5436 DB Schemes left in UK (7751 in 2006)
\item
  £1.6tr in UK Pension Liabilities

  \begin{itemize}
  \tightlist
  \item
    26\% Schemes have more than £100m Liabilities
  \item
    5\% Schemes have more than £1bn Liabilities
  \end{itemize}
\end{itemize}

\end{frame}

\begin{frame}{DB Pension Scheme Risks}
\protect\hypertarget{db-pension-scheme-risks}{}

\begin{itemize}
\tightlist
\item
  Interest Rate
\item
  Inflation
\item
  Longevity
\end{itemize}

\end{frame}

\begin{frame}{History of Pension Fund Hedging}
\protect\hypertarget{history-of-pension-fund-hedging}{}

\begin{itemize}
\tightlist
\item
  Pre - 2000: Immunization:(Matched Duration, Asset Convexity
  \textgreater{} Liability Convexity)

  \begin{itemize}
  \tightlist
  \item
    ignores cost of convexity
  \item
    assumes parallel curve moves
  \end{itemize}
\item
  2000 - 2020: PV01, IE01 buckets across curve

  \begin{itemize}
  \tightlist
  \item
    ignores intra bucket volatility
  \item
    hedge is more complex
  \end{itemize}
\end{itemize}

\end{frame}

\begin{frame}{Current Approaches}
\protect\hypertarget{current-approaches}{}

\begin{itemize}
\tightlist
\item
  Using the whole curve to discount liabilities

  \begin{itemize}
  \tightlist
  \item
    Spurious accuracy
  \end{itemize}
\item
  Picking representative points on the curve

  \begin{itemize}
  \tightlist
  \item
    Choice of points is arbitrary
  \end{itemize}
\item
  Liability Benchmark Portfolio (LBP)

  \begin{itemize}
  \tightlist
  \item
    Convexity mismatch
  \item
    Credit spreads
  \end{itemize}
\end{itemize}

\end{frame}

\begin{frame}{PCA Approach}
\protect\hypertarget{pca-approach}{}

\begin{itemize}
\tightlist
\item
  Assume that linear factor structure exists
\item
  Assume factor loadings are stable
\item
  Infer factors from data
\end{itemize}

\end{frame}

\begin{frame}{UK Yield Curve}
\protect\hypertarget{uk-yield-curve}{}

\includegraphics{UK-Liabs-Pres_files/figure-beamer/chgYieldCurve-1.pdf}

\end{frame}

\begin{frame}{Yield Correlation}
\protect\hypertarget{yield-correlation}{}

\includegraphics{UK-Liabs-Pres_files/figure-beamer/exploredata-1.pdf}

\begin{itemize}
\tightlist
\item
  Darkest on diagonal and lightest in top right corner
\item
  Adjacent maturities should have higher correlations
\end{itemize}

\end{frame}

\begin{frame}{Interpretting the PCAs}
\protect\hypertarget{interpretting-the-pcas}{}

\includegraphics{UK-Liabs-Pres_files/figure-beamer/factorLoadings-1.pdf}

\end{frame}

\begin{frame}{Principal Component Analysis}
\protect\hypertarget{principal-component-analysis}{}

\includegraphics{UK-Liabs-Pres_files/figure-beamer/explainedRisk-1.pdf}

\begin{longtable}[]{@{}lll@{}}
\toprule
PCA 1 & PCA 2 & PCA 3\tabularnewline
\midrule
\endhead
99.6 & 0.35 & 0.04\tabularnewline
\bottomrule
\end{longtable}

\end{frame}

\begin{frame}{Typical UK DB Scheme}
\protect\hypertarget{typical-uk-db-scheme}{}

\includegraphics{UK-Liabs-Pres_files/figure-beamer/liabCashFlowSimulation-1.pdf}

\end{frame}

\begin{frame}{Building the hedge}
\protect\hypertarget{building-the-hedge}{}

\begin{itemize}
\tightlist
\item
  We know the factor loadings of the Scheme
\item
  We know the factor loadings of each point on the yield curve
  \(\implies\) we know factor loading of each bond/swap
\item
  Back out how much of each bond/swap we need to hedge the Scheme
\end{itemize}

\end{frame}

\begin{frame}{How have PCAs behaved}
\protect\hypertarget{how-have-pcas-behaved}{}

\includegraphics{UK-Liabs-Pres_files/figure-beamer/PCAs-1.pdf}

\end{frame}

\begin{frame}{Trade Ideas}
\protect\hypertarget{trade-ideas}{}

As at 2020-09-30

\begin{longtable}[]{@{}llll@{}}
\toprule
. & PCA 1 & PCA 2 & PCA 3\tabularnewline
\midrule
\endhead
Value & -6.231 & -1.116 & -0.433\tabularnewline
SDs from Mean & 2.1 & -1.7 & -2.2\tabularnewline
\bottomrule
\end{longtable}

\end{frame}

\begin{frame}{Selecting an Asset to trade}
\protect\hypertarget{selecting-an-asset-to-trade}{}

\begin{itemize}
\tightlist
\item
  Assume that you want to trade PCA2
\end{itemize}

\includegraphics{UK-Liabs-Pres_files/figure-beamer/showdrift-1.pdf}

\end{frame}

\begin{frame}{Forecasting}
\protect\hypertarget{forecasting}{}

\begin{itemize}
\tightlist
\item
  Idea 1: Fit Orstein Uhlenbeck (Vasicek) Process

  \begin{itemize}
  \tightlist
  \item
    Huggins and Schaller approach
  \item
    assumes the only information that is useful in calculating the
    change is the current value.
  \item
    no trending away from mean
  \end{itemize}
\item
  Idea 2: Fit Auto Regressive model, AR(p)

  \begin{itemize}
  \tightlist
  \item
    somewhat of a Blackbox
  \item
    allows for trending away from mean
  \item
    need to infer the order p, of the AR model
  \end{itemize}
\end{itemize}

\end{frame}

\begin{frame}{AR model}
\protect\hypertarget{ar-model}{}

This chart shows the expected path of the PCA 2. The chart below fits an
AR model.
\includegraphics{UK-Liabs-Pres_files/figure-beamer/showfuturepathDemo,-1.pdf}

\end{frame}

\begin{frame}{Vasicek Model}
\protect\hypertarget{vasicek-model}{}

SDE for Vasicek model : \(dr_{t}= \kappa (\theta - r_t) + \sigma dW_t\)
\includegraphics{UK-Liabs-Pres_files/figure-beamer/fitVasicekmodel-1.pdf}

\end{frame}

\begin{frame}{Conclusions}
\protect\hypertarget{conclusions}{}

\begin{itemize}
\tightlist
\item
  Benefits

  \begin{itemize}
  \tightlist
  \item
    We can simplify the hedge
  \item
    We can identify tactical trading opportunities
  \item
    Challenge trade ideas after execution systematically
  \end{itemize}
\item
  Limitations/Further Research

  \begin{itemize}
  \tightlist
  \item
    explaining approach to Trustees
  \item
    stability of factor loadings
  \item
    Converting ZC to real bonds
  \end{itemize}
\end{itemize}

\end{frame}

\begin{frame}{Dataset}
\protect\hypertarget{dataset}{}

Source: Bank of England daily nominal Zero Coupon Yields estimated using
Nelsen Siebold Method.
\url{https://www.bankofengland.co.uk/-/media/boe/files/statistics/yield-curves/latest-yield-curve-data.zip}

\end{frame}

\end{document}
